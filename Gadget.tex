\section{Gadget Matrices and Gadget-Based Matrix Operations}
In this section, we will show that the matrix $\G$ defined above can be used to support "invariant-preserving" pseudo-commutative matrix multiplication operations. We can also generalize matrix $\G$ and its trapdoor to other integer powers or mixed-integer products. lastly, we propose a new arrangement of matrix operations for the generalized gadget matrices that is critical to our main result.
\subsection{Matrix Operations with G via the Flattening Function $\G^{-1}$}
Over general lattices, the only matrices that commute with $\G \in \Z^{n \times m}_{q}$ are scaled identity matrices $\alpha\mat{I}$, in the sense that $\G.(\alpha\mat{I}_{m})=(\alpha\mat{I}_{n}).\G$. Note that if the $\G$ is padded, then $\alpha\mat{I}_{m}$ could alternatively be the block matrix $\mat{A}$ containing $\alpha\mat{I}_{n}$ in the appropriate with zeroes everywhere else.\

The works of Xagawa[.....] and Alperin-Sheriff and Peikert[......], among others, describe a technique that resolves this non-commutatively problem for $\G$. In particular, there is an efficiently computable function $\G^{-1}$, so that for any matrix $\mat{B} \in \Z^{n \times m}_{q}$��so that $\G^{-1}(\mat{B})=\mat{X} \in \{-1,1\}^{m \times m}$ and $\mat{GX}=\mat{B}$. This allows for "pseudo-commutative" multiplication of the gadget matrix $\G$ by any square matrix $\mat{M}$ of dimension $n$, by observing that
\begin{equation}
\G.(\G^{-1}(\mat{MG}))=\mat{M.G}
\end{equation}
The matrix $\mat{X}=\G^{-1}(\mat{B})$ has small norm independent of $\mat{B}$.
\subsection{Non-Binary Gadgets $\G_{n,r,m}$ and Function $\G^{-1}_{n^{'},r^{'},m^{'}}(.)$}
The matrix $\G$ and its trapdoor can be extended to other integer powers or mixed-integer products by the results of [MP......]. Then, we can give a generalized notation for gadget matrices as follows:\

For any modulus $q\geq 2$, for integer $2\leq r\leq q$, let $\mat{g}^{T}_{r}=[1,r,r^{2},...,r^{k_{r}-1}] \in \Z^{1 \times k_{r}}_{q}$ for $k_{r}=\lceil \log_{r}q \rceil$. We let $\G_{n,r}=\mat{g}^{T}_{r}\otimes \mat{I}_{n} \in \Z^{n \times nk_{r}}_{q}$. The public trapdoor basis can be given analogously. Similar to the above padding argument, $\G_{n,r} \in \Z^{n \times nk_{r}}_{q}$ can be padded into a matrix $\G_{n,r,m} \in \Z^{n \times m}_{q}$ for $m\geq nk_{r}$ without increasing the norm of $\widetilde{\mat{T}_{\G_{n,r,m}}}$ from that of $\widetilde{\mat{T}_{\G_{n,r}}}$.\

In this paper, we do not need to use $\widetilde{\mat{T}_{\G_{n,r,m}}}$ or $\widetilde{\mat{T}_{\G_{n,r}}}$ at all, but keep the discussion for exposition. Under this notation, the matrix $\G$ in dimension $n$ is either $\G_{n,2} \in \Z^{n \times n\log_{2}q}_{q}$ or its padded version $\G_{n,2,m} \in \Z^{n \times m}_{q}$ depending on the setting.\

We now introduce a related function - the Batch Change-of-Base function $\G^{-1}_{n^{'},r^{'},m^{'}}(.)$ - as follows:\

For any modulus $q \geq 2$, and any integer base $2 \leq r^{'} \leq q$, let integer $k_{r^{'}}=\lceil \log_{r^{'}}q \rceil$. For any integer $N^{'} \geq 2$ and $m^{'} \geq n^{'}k_{r^{'}}$ the function $\G^{-1}_{n^{'},r^{'},m^{'}}(.)$ takes as input a matrix from $\Z^{n^{'} \times m^{'}}_{q}$, first computes a matrix in $\{0,1,...,r^{'}-1\}^{n^{'}\log_{r^{'}}q \times m^{'}}$ using the $\G^{-1}$, then pads with rows of zeroes needed to form a matrix in $\{0,1,...,r^{'}-1\}^{m^{'} \times m^{'}}$. For example, the typical base-2 $\G^{-1}=\G^{-1}_{n,2,m}$ takes $\Z^{n \times m}_{q}$ to $\{0,1\}^{m \times m}$ as expected.
\subsection{Further Gadget-Based Matrix Multiplication Operations}
In this part, we propose a new arrangement of matrix operations(pseudo-commutative and non-commutative) that is critical to our main construction in this paper. First, fix any integer $n$, for some sufficiently small $l=\omega(1)\ll \log_{2}q \approx \log_{2}n$, define $l^{'}=2^{l}$ such that $l^{'}= \omega(1)< \log_{2}q \approx \log_{2}n$. Finally, fix any integer $m\geq nlk_{l^{'}}$.\

Then for any two matrices $\mat{H} \in \Z^{n \times nl}_{q}, \mat{X} \in \Z^{ln \times n}_{q}$, consider the following terms, in order:
\begin{enumerate}
\item First, consider the base-2, dimension-n "gadget-encoding" of $\mat{X} \in \Z^{ln \times n}_{q}$, i.e. the matrix
\begin{equation}
\textbf{X}.\G_{n,2,m} \in \Z^{ln \times m}_{q}=\Z^{n\log_{2}(l^{'}) \times m}.
\end{equation}
\item Next, consider the base-$l^{'}$, dimension-$(nl)$ flatting(with zero-row padding) of the above:
\begin{equation}
\G^{-1}_{nl,l^{'},m}(\mat{X}.\G_{n,2,m}) \in \{0,1,...,l^{'}-1\}^{m \times m} \subsetneq \Z^{m \times m}_{q}.
\end{equation}
\item Then, consider the base-$l^{'}$, dimension-$(nl)$ gadget-encoding of $\mat{H} \in \Z^{n \times nl}_{q}$, i.e. the matrix
\begin{equation}
\mat{H}.\G_{nl,l^{'},m} \in \Z^{n \times m}_{q}.
\end{equation}
\item Finally - for sufficient large $m$, we find the following relationship holds:
\begin{equation}
(\mat{H}.\G_{nl,l^{'},m}).(\G^{-1}_{nl,l^{'},m}(\mat{X}.\G_{n,2,m}))=(\mat{H.X}).\G_{n,2,m} \in \Z^{n \times m}_{q}
\end{equation}
\end{enumerate}
with $\| \G^{-1}_{nl,l^{'},m}(\mat{X}.\G_{n,2,m}) \|= \textbf{small}$, conditioned on the sufficiently small choice of $l=\omega(1)$. We emphasize that only public information, $n,m,l$, is required to perform this batch base-change-then-multiply operation, when given as input any $\mat{M} \in \Z^{n \times m}_{q}$(equal to $\mat{H}.\G_{nl,l^{'},m} \in \Z^{n \times m}_{q}$) and any $\mat{X} \in \Z^{ln \times n}_{q}$.
\begin{definition}
We refer to the matrix $\mat{H}.\G_{nl,l^{'},m} \in \Z^{n \times m}_{q}$ as the predicate-encoding of $\mat{H} \in \Z^{n \times nl}_{q}$, and to the matrix $\G^{-1}_{nl,l^{'},m}(\mat{X}.\G_{n,2,m}) \in \{0,1,...,l^{'}-1\}^{m \times m} \subsetneq \Z^{m \times m}_{q}$ as the input-encoding of $\mat{X} \in \Z^{ln \times n}_{q}$.
\end{definition}









