\documentclass[11pt,hidelinks]{article}
\usepackage{url}
\usepackage{times}
\usepackage[margin=1in]{geometry}
\usepackage{amssymb,amsmath,amsfonts,amsthm}
\usepackage{enumerate}
\usepackage{enumitem}
\usepackage{framed}
\usepackage{xcolor,xspace}
\usepackage{hyperref}
\usepackage{float,subfigure}
\newcommand{\N}{\ensuremath{\mathbb{N}}}
\newcommand{\Z}{\ensuremath{\mathbb{Z}}}
\newcommand{\R}{\ensuremath{\mathbb{R}}}
\newcommand{\Q}{\ensuremath{\mathbb{Q}}}
\newcommand{\poly}{\mathsf{poly}}
\newcommand{\prob}{\mathsf{Pr}}
\renewcommand\vec[1]{\ensuremath\boldsymbol{#1}}
\newcommand{\bool}{\{0,1\}}
\newcommand{\minusplus}{\{-1,1\}}
\newcommand{\xor}{\oplus}
\newcommand{\D}{\mathcal{D}}
\renewcommand{\S}{\mathcal{S}}
\newcommand{\mat}[1]{\mathbf{#1}}%
\newcommand{\T}{\mathbf{T}}



\newcommand{\SIS}{\ensuremath{\mathsf{SIS}}}
\newcommand{\LWE}{\ensuremath{\mathsf{LWE}}}
\newcommand{\RLWE}{\ensuremath{\mathsf{RLWE}}}
\newcommand{\Gapsvp}{\ensuremath{\mathsf{GapSVP}}}
\newcommand{\Sivp}{\ensuremath{\mathsf{SIVP}}}
\newcommand{\sampleleft}{\ensuremath{\mathsf{SampleLeft}}\xspace}
\newcommand{\sampleright}{\ensuremath{\mathsf{SampleRight}}\xspace}
\newcommand{\samplepre}{\ensuremath{\mathsf{SamplePre}}\xspace}
\newcommand{\trapgen}{\ensuremath{\mathsf{TrapGen}}\xspace}
\newcommand{\G}{\ensuremath{\mathbf{G}}}

\newcommand{\Gr}{\ensuremath{\mathbb{G}}}
\newcommand{\F}{\ensuremath{\mathbb{F}}}

\newcommand{\ignore}[1]{}
\newcommand{\etal}{et al.\ }
\newcommand{\secparam}{\ensuremath{\lambda}\xspace}
\newcommand{\A}{\ensuremath{\mathcal{A}}\xspace}
\newcommand{\B}{\ensuremath{\mathcal{B}}\xspace}
\renewcommand{\O}{\ensuremath{\mathcal{O}}\xspace}
\newcommand{\Expt}{\ensuremath{\mathbf{Expt}}\xspace}
\newcommand{\ppt}{\ensuremath{\textsc{ppt}}\xspace}
\newcommand{\negl}{\ensuremath{\mathsf{negl}}\xspace}
\renewcommand{\H}{\ensuremath{\mathcal{H}}\xspace}
\newcommand{\X}{\ensuremath{\mathcal{X}}\xspace}
\newcommand{\Y}{\ensuremath{\mathcal{Y}}\xspace}
\newcommand{\hybrid}{\ensuremath{\mathsf{H}}\xspace}


\newcommand{\pk}{\ensuremath{\mathsf{pk}}\xspace}
\newcommand{\sk}{\ensuremath{\mathsf{sk}}\xspace}
\newcommand{\vk}{\ensuremath{\mathsf{vk}}\xspace}
\newcommand{\mpk}{\ensuremath{\mathsf{mpk}}\xspace}
\newcommand{\msk}{\ensuremath{\mathsf{msk}}\xspace}
\newcommand{\pp}{\ensuremath{\mathsf{pp}}\xspace}
\newcommand{\keygen}{\ensuremath{\mathsf{KeyGen}}\xspace}
\newcommand{\enc}{\ensuremath{\mathsf{Enc}}\xspace}
\newcommand{\dec}{\ensuremath{\mathsf{Dec}}\xspace}
\newcommand{\delegate}{\ensuremath{\mathsf{Delegate}}\xspace}
\newcommand{\setup}{\ensuremath{\mathsf{Setup}}\xspace}
\newcommand{\extract}{\ensuremath{\mathsf{Extract}}\xspace}
\newcommand{\fuzzy}{\ensuremath{\mathsf{Fuzzy}}\xspace}
\newcommand{\si}{\ensuremath{\mathsf{Sim}}\xspace}
\newcommand{\ct}{\ensuremath{\mathsf{ct}}\xspace}
\newcommand{\iO}{\ensuremath{i\mathcal{O}}\xspace}
\newcommand{\id}{\ensuremath{\mathsf{id}}\xspace}
\newcommand{\ID}{\ensuremath{\mathsf{ID}}\xspace}
\newcommand{\advantage}{\mathbf{Adv}}

\newcommand{\sign}{\ensuremath{\mathsf{Sign}}\xspace}
\newcommand{\verify}{\ensuremath{\mathsf{Verify}}\xspace}


\newcommand{\IBE}{\ensuremath{\mathsf{IBE}}\xspace}
\newcommand{\ABE}{\ensuremath{\mathsf{ABE}}\xspace}
\newcommand{\IPE}{\ensuremath{\mathsf{IPE}}\xspace}
\newcommand{\FE}{\ensuremath{\mathsf{FE}}\xspace}
\newcommand{\PE}{\ensuremath{\mathsf{PE}}\xspace}


\newtheorem*{thm*}{Theorem}
\newtheorem{theorem}{Theorem}[section]
\newtheorem{definition}[theorem]{Definition}
\newtheorem{remark}[theorem]{Remark}
\newtheorem{lemma}[theorem]{Lemma}
\newtheorem{claim}[theorem]{Claim}
\newtheorem{cor}[theorem]{Corollary}
\newtheorem{asmp}[theorem]{Assumption}
\newtheorem{proposition}[theorem]{Proposition}
\newtheorem{fact}[theorem]{Fact}




\newcommand{\leo}[1]{{\color{cyan}[Leo: #1]}}
\newcommand{\todo}[1]{{\color{red}[TO-DO: #1]}}
\newcommand{\proofsketch}{ \noindent {\it Proof (sketch): }}
\newcommand{\sampled}{\mathsf{SampleD}}


\begin{document}
\title{Compact Inner Product Encryption from LWE}%\\

\author{}
\date{}
\maketitle

\begin{abstract}
s\\
s\\
\end{abstract}




\section{Introduction}
Functional encryption has become quite popular in last few years because it provides the system administrator with fine-grained control over the decryption capabilities of its users. Two important examples of functional encryption are $attirbute-based\ encryption(\ABE)$[......] and $predicate\ encryption(\PE)$[......]. 
In (key-police)$\ABE$ and $\PE$ systems, each ciphertext $c$ is associated with an attribute $a$ and each secret key $s$ is associated with predicate $f$. A user holding the key $s$ can decrypt $c$ if and only if $f(a)=1$. 
The difference between the two types of systems is in amount of information revealed: an $\ABE$ system reveals the attribute associated with each ciphertext, while a $\PE$ system keeps the attribute hidden.\

The most expressive $\PE$ scheme is that of Katz, Sahai, and Waters[KSW08......], which is a $\PE$ scheme that supports the inner-product predicate: attributes $a$ and predicates $f$ are expressed as vectors $\overrightarrow{v}_{a}$ and $\overrightarrow{w}_{f}$. 
We say $f(a)=1$ if and only if $\langle \overrightarrow{v}_{a}, \overrightarrow{w}_{f} \rangle=0$. They also showed that inner product predicates can support conjunction, subset and range queries on encrypted date[......] as well as disjunctions, polynomial evaluation, and $\mat{CNF}$ and $\mat{DNF}$ formulas[......].\

In recent years, there were a number of IPE schemes[KSW08, OT09, LOS$^{+}$10, OT10, AL10, Par11, OT11, OT12] have been proposed, the security of these schemes is based on composite-number or prime order groups. 
while assumption such as these can often be shown to hold in a suitable generic group model, to obtain more confidence in security, we would like to build IPE scheme based on computational problems such as lattice-based hardness problems, whose complexity is better understood.\

Agrawal, Freeman, and Vaikuntanathan[AFV11] proposed the first IPE scheme based on the LWE assumption, and Xagawa[Xag13] improved the efficiency of [AFV11]'s scheme. 
While the scheme of [Xag13] has public parameters of size $O(l n^{2}\lg^{2}q)$ and ciphertexts of size $O(l n\lg^{2}q)$, where $l$ is the length of predicate vector, $n$ is the security parameter, and $q$ is the modulus. which still makes the scheme impractical.

\subsection{Our Contributions}

\subsection{Related Work} 
\section{Preliminaries}
\paragraph{Notation.} Let $\secparam$ be the security parameter, and let $\ppt$ denote probabilistic polynomial time. We use bold uppercase letters to denote matrices ${\bf M}$, and bold lowercase letters to denote vectors $\vec{v}$. We write $\widetilde{\mat M}$ to denote the Gram-Schmidt orthogonalization of $\mat M.$ We write $[n]$ to denote the set $\{1,...,n\}$, and $|\vec{t}|$ to denote the number of bits in the string $\vec{t}$. We denote the $i$-th bit $\vec{s}$ by $\vec{s}[i]$. We say a function $\negl(\cdot): \N \rightarrow (0,1)$ is negligible, if for every constant $c \in \N$, $\negl(n) < n^{-c}$ for sufficiently large $n$.

\subsection{Inner Product Encryption}
We recall the syntax and security definition of \emph{inner product encryption} (IPE). IPE can be regarded as a generalization of predicate encryption~\cite{C:Shamir84,C:BonFra01}. An IPE scheme $\Pi$ consists of the 4-tuple $(\setup, \keygen, \enc,
\\
\dec)$ algorithms with details as follows:
\begin{description}
 \item $\setup(1^\secparam)$: On input the security parameter $\secparam$, the setup algorithm outputs public parameters $\pp$ and master secret key $\msk$.
 \item $\keygen(\msk, \vec{v})$: On input the master secret key $\msk$ and a predicate vector $\vec{v}$, the key generation algorithm outputs a secret key $\sk_{\vec{v}}$ for vector $\vec{v}$.
 \item $\enc(\pp, \vec{w}, \mu)$: On input the public parameter $\pp$ and an attribute/message pair $(\vec{w}, \mu)$, it outputs a ciphertext $\ct_{\vec{w}}$.
 \item $\dec(\sk_{\vec{v}}, c_{\vec{w}})$: On input the secret key $\sk_{\vec{v}}$ and a ciphertext $c_{\vec{w}}$, it outputs the corresponding plaintext $\mu$ if $\langle \vec{v}, \vec{w} \rangle = 0$; otherwise, it outputs $\bot$.
\end{description}

\paragraph{Correctness.} We say the IPE scheme described above is correct, if for any $(\msk, \pp) \leftarrow \setup(1^\secparam)$, any message $\mu$, predicate vector $\vec{v}$, and any attribute vector $\vec{w}$ where $\langle \vec{v}, \vec{w}\rangle = 0$, we have $\dec(\sk_{\vec{v}}, \ct_{\vec{w}}) = \mu$, where $\sk_{\vec{w}} \leftarrow \keygen(\msk, \vec{v})$ and $\ct_{\vec{v}} \leftarrow \enc(\pp, \vec{w}, \mu)$.

\paragraph{Security.}For the security definition of IPE, we use the following experiment to describe it. Formally, for any $\ppt$ adversary $\A$, we consider the experiment $\Expt_{\A}^{\mathsf{IPE}}(1^\secparam)$:
\begin{itemize}
 \item \textbf{Setup}: An adversary $\A$ outputs two attribute vectors $\vec{w}_{0}, \vec{w}_{1}$,  A challenger runs the $\setup(1^\secparam)$ algorithm, and sends the master public key $\pp$ to the adversary.
 \item \textbf{Query Phase I}: Proceeding adaptively, the adversary $\A$ queries a sequence of predicate vectors $(\vec{v}_1,..., \vec{v}_m)$ subject to the restriction that $\langle \vec{v}_i, \vec{w}_{0} \rangle \neq 0$ and $\langle \vec{v}_i, \vec{w}_{1} \rangle \neq 0$. On the $i$-th query, the challenger runs $\keygen(\msk, \vec{v}_i),$ and sends the result $\sk_{\vec{v}_i}$ to $\A$.
 \item \textbf{Challenge}: Once adversary $\A$ decides that Query Phase I is over, it outputs  two length-equal messages $(\mu^*_0, \mu^*_1)$ and send them to challenger.  In response, the challenger selects random $b \in \bool$, and sends the ciphertext $\enc(\pp, \vec{w}_{b}, \mu^*_b)$ to \A.
 \item \textbf{Query Phase II}: Adversary $\A$ continues to issue identity queries $(\vec{v}_{m + 1},..., \vec{v}_{n})$ adaptively, under the restriction that $\langle \vec{v}_i, \vec{w}_{0} \rangle \neq 0$ and $\langle \vec{v}_i, \vec{w}_{1} \rangle \neq 0$. The challenger responds by issuing keys $\sk_{\vec{v}_i}$ as in Query Phase I.
 \item \textbf{Guess}: Adversary $\A$ outputs a guess $b' \in \bool$.
\end{itemize}
We define the advantage of adversary $\A$ in attacking an IPE scheme $\Pi$ as
$$\advantage_{\A}(1^\secparam) = \left|\Pr[b = b'] - \frac{1}{2}\right|,$$
\noindent where the probability is over the randomness of the challenger and adversary.

\begin{definition}\label{defn:sec}
We say an IPE scheme $\Pi$ is weakly attribute hiding against chosen-plaintext attacks in selective attribute setting, if for all $\ppt$ adversaries $\A$, we have
$$\advantage_{\A}(1^\secparam) \leq \negl(\secparam).$$
\end{definition}


\subsection{Lattice Background}
A full-rank $m$-dimensional integer lattice $\Lambda\subset\Z^m$ is a discrete additive subgroup whose linear span is $\R^m$. The basis of $\Lambda$ is a linearly independent set of vectors whose linear combinations are exactly $\Lambda$. Every integer lattice is generated as the $\Z$-linear combination of linearly independent vectors $\mat{B}=\{\vec{b}_1,...,\vec{b}_m\}\subset\Z^m$. For a matrix $\mat{A}\in\Z^{n\times m}_q$, we define the ``$q$-ary'' integer lattices:
$$\Lambda_q^{\bot}=\{\vec{e} \in \Z^m| \mat{A} \vec{e} = \vec{0} \bmod q\},\qquad  \Lambda_q^{\mat{u}}=\{\vec{e}\in\Z^m|\mat{A}\vec{e} = \vec{u} \bmod q\}$$
It is obvious that $\Lambda_q^{\vec{u}}$ is a coset of $\Lambda_q^{\bot}$.

Let $\Lambda$ be a discrete subset of $\Z^m$. For any vector $\vec{c}\in\R^m$, and any positive parameter $\sigma\in\R$, let $\rho_{\sigma, \vec{c}}(\vec{x})=\exp(-\pi||\vec{x}-\vec{c}||^2 / \sigma^2)$ be the Gaussian function on $\R^m$ with center $\vec{c}$ and parameter $\sigma$. 
Next, we let $\rho_{\sigma, \vec{c}}(\Lambda)=\sum_{\vec{x}\in\Lambda}\rho_{\sigma, \vec{c}}(\vec{x})$ be the discrete integral of $\rho_{\sigma, \vec{x}}$ over $\Lambda,$ and let $\D_{\Lambda, \sigma, \vec{c}}(\vec{y}):=\frac{\rho_{\sigma, \vec{c}}(\vec{y})}{\rho_{\sigma, \vec{c}}(\Lambda)}$. 
We abbreviate this as $\D_{\Lambda, \sigma}$ when $\vec{c}=\vec{0}.$

Let $S^m$ denote the set of vectors in $\R^{m + 1}$ whose length is 1. Then the norm of a matrix $\mat{R} \in \R^{m \times m}$ is defined to be $\mathsf{sup}_{\vec{x} \in S^m} ||\mat{R} \vec{x}||$. Then we have the following lemma, which bounds the norm for some specified distributions.

\begin{lemma}[\cite{{EC:AgrBonBoy10}}]\label{lem:bound}
Regarding the norm defined above, we have the following bounds:
\begin{itemize}
 \item Let $\mat{R} \in \{-1, 1\}^{m \times m}$ be chosen at random, then we have $\prob[||\mat{R}|| > 12 \sqrt{2m}] < e^{-2m}$.
 \item Let $\mat{R}$ be sampled from $\D_{\Z^{m \times m}, \sigma}$, then we have $\prob [||\mat{R}|| > \sigma \sqrt{m}] < e^{-2m}$.
\end{itemize}
\end{lemma}

\subsection{Randomness Extraction}

We will use the following lemma to argue the indistinghishability of two different distributions, which is a generalization of the leftover hash lemma proposed by Dodis et al. \cite{EC:DodReySmi04}.

\begin{lemma}[\cite{EC:AgrBonBoy10}] \label{lem:lhl}
Suppose that $m > (n + 1) \log q + \omega(\log n)$. Let $\mat{R} \in \{-1, 1\}^{m \times k}$ be chosen uniformly at random for some polynomial $k = k(n)$. Let $\mat{A}, \mat{B}$ be matrix chosen randomly from $\Z^{n \times m}_q, \Z^{n \times k}_q$ respectively. Then, for all vectors $\vec{w} \in \Z^m$, the two following distributions are statistically close:
$$(\mat{A}, \mat{A} \mat{R}, \mat{R}^T \vec{w}) \approx (\mat{A}, \mat{B}, \mat{R}^T \vec{w})$$
\end{lemma}

\subsection{Learning With Errors}

The LWE problem was introduced by Regev \cite{STOC:Regev05}, who showed that solving it \emph{on average} is
as hard as (quantumly) solving several standard lattice problems \emph{in the worst case}.
\begin{definition}[LWE]\label{defn:lwe}
For an integer $q = q(n) \geq 2$, and an error distribution $\chi = \chi(n)$ over $\Z_q$, the \emph{Learning With Errors problem $\LWE_{n, m, q, \chi}$} is to distinguish between the following pairs of distributions (e.g. as given by a sampling oracle $\mathcal{O}\in\{\mathcal{O}_{\vec{s}}, \mathcal{O}_{\$}\}$):
$$\{\mat{A}, \mat{A}\vec{s} + \vec{x}\} \  \text{and} \ \{\mat{A}, \vec{u}\}$$
where $\mat{A} \overset{\$}{\leftarrow}\Z^{n \times m}_q$, $\vec{s} \overset{\$}{\leftarrow} \Z^n_q$, $\vec{u} \overset{\$}{\leftarrow} \Z^m_q$, and $\vec{x} \overset{\$}{\leftarrow} \chi^n$.
\end{definition}


\subsection{Two-Sided Trapdoors and Sampling Algorithms}

We will use the following algorithms to sample short vectors from specified lattices.

\begin{lemma}[\cite{STOC:GenPeiVai08,Alwen2010}] \label{lem:trapgen}
Let $q, n, m$ be positive integers with $q\geq 2$ and sufficiently large $m = \Omega(n \log q)$. There exists a $\ppt$ algorithm $\trapgen(q, n, m)$ that with overwhelming probability outputs a pair $(\mat{A}\in\Z_q^{n\times m}, \mat{T}_\mat{A} \in\Z^{m\times m})$ such that $\mat{A}$ is statistically close to uniform in $\Z_q^{n\times m}$ and $\mat{T}_\mat{A}$ is a basis for $\Lambda_q^{\bot}(\mat{A})$ satisfying
$$||\mat{T}_\mat{A}||\leq O(n\log q)\quad\mbox{and}\quad||\widetilde{\mat{T}_{\mat{A}}}||\leq O(\sqrt{n\log q})$$
except with $\mathsf{negl}(n)$ probability.
\end{lemma}

\begin{lemma}[\cite{STOC:GenPeiVai08,EC:CHKP10,EC:AgrBonBoy10}] \label{lem:samp}
Let $q>2, m>n.$ There are three algorithms as follows:
\begin{itemize}
 \item There is a \ppt\ algorithm $\samplepre(\mat{A}, \mat{T}_{\mat{A}}, \vec{u}, \sigma)$: It takes as input
\begin{itemize}
\item a rank-$n$ matrix $\mat{A}\in\Z_q^{n\times m},$ and a vector $\vec{u}\in\Z_q^n,$
\item a ``short'' basis $\mat{T}_{\mat{A}}$ for lattice $\Lambda_q^{\bot}(\mat{A}),$ and a Gaussian parameter $\sigma > ||\widetilde{\mat{T}_{\mat{A}}}||\cdot\omega(\sqrt{\log m}),$
\end{itemize}
then outputs a vector $\vec{x} \in \Lambda^{\perp}_{q}(\mat{A})$ distributed statistically close to $\mathcal{D}_{\Lambda^{\vec{u}}_{q}(\mat{A}),\sigma}$, whenever $\Lambda^{\vec{u}}_{q}(\mat{A})$ is not empty.
\end{itemize}

\begin{itemize}
 \item There is a \ppt\ algorithm $\sampleleft(\mat{A}, \mat{B}, \mat{T}_{\mat{A}}, \vec{u}, s)$: It takes as input
\begin{itemize}
\item a rank-$n$ matrix $\mat{A}\in\Z_q^{n\times m},$ and any matrix $\mat{B}\in\Z_q^{n\times m_1},$
\item a ``short'' basis $\mat{T}_{\mat{A}}$ for lattice $\Lambda_q^{\bot}(\mat{A}),$ a vector $\vec{u}\in\Z_q^n,$
\item and a Gaussian parameter $s > ||\widetilde{\mat{T}_{\mat{A}}}||\cdot\omega(\sqrt{\log(m+m_1)}),$
\end{itemize}
then outputs a vector $\vec r\in\Z^{m+m_1}$ distributed statistically close to $\D_{\Lambda_q^{\vec{u}}(\mat{F}), s}$ where $\mat{F}:=(\mat{A}|\mat{B}).$

 \item There is a \ppt\ algorithm $\sampleright(\mat{A}, \mat{B}, \mat{R}, \mat{T}_{\mat{B}}, \vec{u}, s)$: It takes as input
\begin{itemize}
\item any matrix $\mat{A}\in\Z_q^{n \times m},$ and a rank-$n$ matrix $\mat{B}\in\Z_q^{n\times m},$
\item a matrix $\mat{R}\in\Z_q^{m \times m},$ where $s_{\mat R} := ||\mat{R}|| = \sup_{\vec{x} : ||\vec{x}||=1}||\mat{R}\vec{x}||,$
\item a ``short'' basis $\mat{T}_{\mat{B}}$ for lattice $\Lambda_q^{\bot}(\mat{B}),$ a vector $\vec{u}\in\Z_q^n,$
\item and a Gaussian parameter $s > ||\widetilde{\mat{T}_{\mat{B}}}||\cdot{s_{\mat R}}\cdot\omega(\sqrt{\log{m}}),$
\end{itemize}
then outputs a vector $\vec{r}\in\Z^{2m}$ distributed statistically close to $\D_{\Lambda_q^{\vec{u}}(\mat{F}), s}$ where $\mat{F}:=(\mat{A}|\mat{A}\mat{R} + \mat{B}).$


\end{itemize}
\end{lemma}




\section{Gadget Matrices and Gadget-Based Matrix Operations}
In this section, we will show that the matrix $\G$ defined above can be used to support "invariant-preserving" pseudo-commutative matrix multiplication operations. We can also generalize matrix $\G$ and its trapdoor to other integer powers or mixed-integer products. lastly, we propose a new arrangement of matrix operations for the generalized gadget matrices that is critical to our main result.
\subsection{Matrix Operations with G via the Flattening Function $\G^{-1}$}
Over general lattices, the only matrices that commute with $\G \in \Z^{n \times m}_{q}$ are scaled identity matrices $\alpha\mat{I}$, in the sense that $\G.(\alpha\mat{I}_{m})=(\alpha\mat{I}_{n}).\G$. Note that if the $\G$ is padded, then $\alpha\mat{I}_{m}$ could alternatively be the block matrix $\mat{A}$ containing $\alpha\mat{I}_{n}$ in the appropriate with zeroes everywhere else.\

The works of Xagawa[.....] and Alperin-Sheriff and Peikert[......], among others, describe a technique that resolves this non-commutatively problem for $\G$. In particular, there is an efficiently computable function $\G^{-1}$, so that for any matrix $\mat{B} \in \Z^{n \times m}_{q}$��so that $\G^{-1}(\mat{B})=\mat{X} \in \{-1,1\}^{m \times m}$ and $\mat{GX}=\mat{B}$. This allows for "pseudo-commutative" multiplication of the gadget matrix $\G$ by any square matrix $\mat{M}$ of dimension $n$, by observing that
\begin{equation}
\G.(\G^{-1}(\mat{MG}))=\mat{M.G}
\end{equation}
The matrix $\mat{X}=\G^{-1}(\mat{B})$ has small norm independent of $\mat{B}$.
\subsection{Non-Binary Gadgets $\G_{n,r,m}$ and Function $\G^{-1}_{n^{'},r^{'},m^{'}}(.)$}
The matrix $\G$ and its trapdoor can be extended to other integer powers or mixed-integer products by the results of [MP......]. Then, we can give a generalized notation for gadget matrices as follows:\

For any modulus $q\geq 2$, for integer $2\leq r\leq q$, let $\mat{g}^{T}_{r}=[1,r,r^{2},...,r^{k_{r}-1}] \in \Z^{1 \times k_{r}}_{q}$ for $k_{r}=\lceil \log_{r}q \rceil$. We let $\G_{n,r}=\mat{g}^{T}_{r}\otimes \mat{I}_{n} \in \Z^{n \times nk_{r}}_{q}$. The public trapdoor basis can be given analogously. Similar to the above padding argument, $\G_{n,r} \in \Z^{n \times nk_{r}}_{q}$ can be padded into a matrix $\G_{n,r,m} \in \Z^{n \times m}_{q}$ for $m\geq nk_{r}$ without increasing the norm of $\widetilde{\mat{T}_{\G_{n,r,m}}}$ from that of $\widetilde{\mat{T}_{\G_{n,r}}}$.\

In this paper, we do not need to use $\widetilde{\mat{T}_{\G_{n,r,m}}}$ or $\widetilde{\mat{T}_{\G_{n,r}}}$ at all, but keep the discussion for exposition. Under this notation, the matrix $\G$ in dimension $n$ is either $\G_{n,2} \in \Z^{n \times n\log_{2}q}_{q}$ or its padded version $\G_{n,2,m} \in \Z^{n \times m}_{q}$ depending on the setting.\

We now introduce a related function - the Batch Change-of-Base function $\G^{-1}_{n^{'},r^{'},m^{'}}(.)$ - as follows:\

For any modulus $q \geq 2$, and any integer base $2 \leq r^{'} \leq q$, let integer $k_{r^{'}}=\lceil \log_{r^{'}}q \rceil$. For any integer $N^{'} \geq 2$ and $m^{'} \geq n^{'}k_{r^{'}}$ the function $\G^{-1}_{n^{'},r^{'},m^{'}}(.)$ takes as input a matrix from $\Z^{n^{'} \times m^{'}}_{q}$, first computes a matrix in $\{0,1,...,r^{'}-1\}^{n^{'}\log_{r^{'}}q \times m^{'}}$ using the $\G^{-1}$, then pads with rows of zeroes needed to form a matrix in $\{0,1,...,r^{'}-1\}^{m^{'} \times m^{'}}$. For example, the typical base-2 $\G^{-1}=\G^{-1}_{n,2,m}$ takes $\Z^{n \times m}_{q}$ to $\{0,1\}^{m \times m}$ as expected.
\subsection{Further Gadget-Based Matrix Multiplication Operations}
In this part, we propose a new arrangement of matrix operations(pseudo-commutative and non-commutative) that is critical to our main construction in this paper. First, fix any integer $n$, for some sufficiently small $l=\omega(1)\ll \log_{2}q \approx \log_{2}n$, define $l^{'}=2^{l}$ such that $l^{'}= \omega(1)< \log_{2}q \approx \log_{2}n$. Finally, fix any integer $m\geq nlk_{l^{'}}$.\

Then for any two matrices $\mat{H} \in \Z^{n \times nl}_{q}, \mat{X} \in \Z^{ln \times n}_{q}$, consider the following terms, in order:
\begin{enumerate}
\item First, consider the base-2, dimension-n "gadget-encoding" of $\mat{X} \in \Z^{ln \times n}_{q}$, i.e. the matrix
\begin{equation}
\textbf{X}.\G_{n,2,m} \in \Z^{ln \times m}_{q}=\Z^{n\log_{2}(l^{'}) \times m}.
\end{equation}
\item Next, consider the base-$l^{'}$, dimension-$(nl)$ flatting(with zero-row padding) of the above:
\begin{equation}
\G^{-1}_{nl,l^{'},m}(\mat{X}.\G_{n,2,m}) \in \{0,1,...,l^{'}-1\}^{m \times m} \subsetneq \Z^{m \times m}_{q}.
\end{equation}
\item Then, consider the base-$l^{'}$, dimension-$(nl)$ gadget-encoding of $\mat{H} \in \Z^{n \times nl}_{q}$, i.e. the matrix
\begin{equation}
\mat{H}.\G_{nl,l^{'},m} \in \Z^{n \times m}_{q}.
\end{equation}
\item Finally - for sufficient large $m$, we find the following relationship holds:
\begin{equation}
(\mat{H}.\G_{nl,l^{'},m}).(\G^{-1}_{nl,l^{'},m}(\mat{X}.\G_{n,2,m}))=(\mat{H.X}).\G_{n,2,m} \in \Z^{n \times m}_{q}
\end{equation}
\end{enumerate}
with $\| \G^{-1}_{nl,l^{'},m}(\mat{X}.\G_{n,2,m}) \|= \textbf{small}$, conditioned on the sufficiently small choice of $l=\omega(1)$. We emphasize that only public information, $n,m,l$, is required to perform this batch base-change-then-multiply operation, when given as input any $\mat{M} \in \Z^{n \times m}_{q}$(equal to $\mat{H}.\G_{nl,l^{'},m} \in \Z^{n \times m}_{q}$) and any $\mat{X} \in \Z^{ln \times n}_{q}$.
\begin{definition}
We refer to the matrix $\mat{H}.\G_{nl,l^{'},m} \in \Z^{n \times m}_{q}$ as the predicate-encoding of $\mat{H} \in \Z^{n \times nl}_{q}$, and to the matrix $\G^{-1}_{nl,l^{'},m}(\mat{X}.\G_{n,2,m}) \in \{0,1,...,l^{'}-1\}^{m \times m} \subsetneq \Z^{m \times m}_{q}$ as the input-encoding of $\mat{X} \in \Z^{ln \times n}_{q}$.
\end{definition}










\section{Scheme}

\subsection{The Construction}
Let $\lambda \in \Z^{+}$ be the security parameter and $l$ be the length of predicate and attribute vectors. Let $q$ and $m$ be positive integers. Let $\sigma$ and $\alpha$ be positive real Gaussian parameters. Define $k=\lfloor \log_{2}q\rfloor$. \\[0.6cm]
$\setup$$(1^{\lambda},1^{l})$: On input a security parameter $\lambda$ and a parameter $l$ denoting the

~~~~length of predicate and attribute vectors, do:
\begin{enumerate}
\item Use the algorithm \trapgen ($1^{\lambda},q,n,m$) to generate a random matrix $\textbf{A} \in \Z^{n \times m}_{q}$ together with a trapdoor $\mat{T}_\mat{A}$.\

\item Choose a random matrix $\mat{B} \in \Z^{n \times m}_{q}$.\

\item Choose a random vector $\vec{u} \in \Z^{n}_{q}$.
\end{enumerate}
~~Output $\pp=(\mat{A}, \mat{B}, \vec{u}),\msk=\mat{T}_\mat{A}$.\\[0.4cm]
$\keygen$$(\pp,\msk,\vec{v})$: On input the public parameters $\pp$, the master secret key

~~~~ $\msk$, and a predicate vector $\vec{v}=(v_{1},...,v_{l}) \in \Z^{l}_{q}$, do:
\begin{enumerate}
\item Define the input-encoding matrices
\begin{equation}
\mat{V}^{'}= \begin{bmatrix}
v_{1} \textbf{I}_{n}\\
v_{2} \textbf{I}_{n}\\
\vdots\\
v_{l} \textbf{I}_{n}
\end{bmatrix} \in \Z^{ln \times n}_{q},~~~ \mat{V}=\G^{-1}_{nl,l^{'},m}(\mat{V}^{'}.\G_{n,2,m}) \in [l^{'}]^{m \times m}.
\end{equation}

\item Define the  matrix $\mat{U}=\mat{BV} \in \Z^{n \times m}_{q}$, $\mat{A}_{\vec{v}}=[\mat{A}|\mat{U}]$.\

\item Using the master secret key $\msk=(\mat{T}_\mat{A},\sigma)$, compute $\vec{r}\leftarrow$ $\sampleleft$ ($\textbf{A}, \mat{T}_\mat{A},\mat{U}, \vec{u},\sigma$). Then $\vec{r}$ is a vector in $\Z^{2m}$ satisfying $\mat{A}_{\vec{v}}.\vec{r}=\vec{u}$ (mod $q$).
\end{enumerate}
~~Output the secret key  $\sk_{\vec{v}}=\vec{r}$.\\[0.4cm]
$\enc$$(\pp,\vec{w},m)$: On input public parameters $\pp$, an attribute vectors $\vec{w}$, and a

~~~~message $m \in \{0,1\}$, do:
\begin{enumerate}
\item Choose a uniformly random $\vec{s}\xleftarrow{\$} \Z^{n}_{q}$.\

\item Choose a noise vector $\vec{e}_{0}\leftarrow D_{\Z^{m}_{q},\alpha}$ and a noise term $e\leftarrow D_{\Z_{q},\alpha}$.\

\item Compute $\vec{c}_{0}=\vec{s}^{T}\mat{A}+\vec{e}^{T}_{0}$.

\item Define the matrix
\begin{equation}
\mat{W}^{'}= \begin{bmatrix}
w_{1} \mat{I}_{n}&w_{2} \mat{I}_{n}&\ldots&w_{l} \mat{I}_{n}
\end{bmatrix} \in \Z^{n \times ln}_{q},~~~ \mat{W}=\mat{W}^{'}.\G_{nl,l^{'},m} \in \Z^{n \times m}_{q}.
\end{equation}
Pick a random matrix $\mat{R}\xleftarrow{\$} \{-1,1\}^{m \times m}$, define error vector $\vec{e}^{T}_{1}=\vec{e}^{T}_{0}\mat{R}$. Set
\begin{equation}
\vec{c}_{1}=\vec{s}^{T}(\mat{B}+\mat{W})+\vec{e}^{T}_{1},~~~c_{2}=\vec{s}^{T}\vec{u}+e+\lfloor \frac{q}{2} \rfloor m.
\end{equation}
\end{enumerate}
~~Output ciphertext  $\ct=(\vec{c}_{0},\vec{c}_{1},c_{2})$.\\[0.4cm]
$\dec$$(\pp,\sk_{\vec{v}},\ct,\vec{v})$: On input public parameters $\pp$ ,a secret key $\sk_{\vec{v}}$ for predicate

~~ vector $\vec{v}$, and a ciphertext $\ct$, do:
\begin{enumerate}
\item Compute the vector $\vec{c}_{v}=\vec{c}_{1}.\mat{V}$.\

\item Let $\vec{c}=[\vec{c}_{0}|\vec{c}_{v}]$.\

\item Compute $ z\leftarrow c_{2}-\vec{c}\vec{r}$(mod $q$).
\end{enumerate}
~~Output $0$ if $|z|<q/4$ and $1$ otherwise.\\[0.2cm]

\subsection{Correctness}
We prove correctness of our scheme as follow.
\begin{lemma}
For certain parameter choices, our scheme is correct.
\end{lemma}
\noindent $Proof$. Recall that the ciphertext $\ct=(\vec{c}_{0},\vec{c}_{1},c_{2})$. For which $\vec{c}_{0}=\vec{s}^{T}\mat{A}+\vec{e}^{T}_{0},  \vec{c}_{1}=\vec{s}^{T}(\mat{B}+\mat{W})+\vec{e}^{T}_{1}, c_{2}=\vec{s}^{T}\vec{u}+e+\lfloor \frac{q}{2} \rfloor m$, so
\begin{equation}
\vec{c}_{v}=\vec{c}_{1}.\mat{V}=\vec{s}^{T}(\mat{B}+\mat{W}).\mat{V}+\vec{e}^{T}_{1}.\mat{V}=\vec{s}^{T}.\mat{B}.\mat{V}+\vec{s}^{T}.\mat{W}.\mat{V}+
\vec{e}^{T}_{1}.\mat{V}.
\end{equation}
It's easy to check that if $\langle \vec{v},\vec{w} \rangle=0$, then $\mat{W}.\mat{V}=\mat{W}^{'}.\mat{V}^{'}.\G_{n,2,m}=\mat{0}$. Then $\vec{c}_{v}=\vec{s}^{T}.\mat{B}.\mat{V}+\vec{e}^{T}_{1}.\mat{V}$, so
\begin{equation}
\begin{aligned}
\vec{c}&=[\vec{c}_{0}|\vec{c}_{v}]=[\vec{s}^{T}\mat{A}+\vec{e}^{T}_{0}|\vec{s}^{T}.\mat{B}.\mat{V}+\vec{e}^{T}_{1}.\mat{V}]=\vec{s}^{T}[\mat{A}|\mat{B}.\mat{V}]+[\vec{e}^{T}_{0}|\vec{e}^{T}_{1}.\mat{V}]\\
&=\vec{s}^{T}[\mat{A}|\mat{U}]+[\vec{e}^{T}_{0}|\vec{e}^{T}_{1}.\mat{V}]=\vec{s}^{T}\mat{A}_{v}+[\vec{e}^{T}_{0}|\vec{e}^{T}_{1}.\mat{V}].
\end{aligned}
\end{equation}
And
\begin{equation}
\begin{aligned}
z&=c_{2}-\vec{c}\vec{r}(mod\ q)=\vec{s}^{T}\vec{u}+e+\lfloor \frac{q}{2} \rfloor m-\vec{s}^{T}\vec{u}-[\vec{e}^{T}_{0}|\vec{e}^{T}_{1}.\mat{V}]\vec{r}\\
&=\lfloor \frac{q}{2} \rfloor m+\underbrace{e-[\vec{e}^{T}_{0}|\vec{e}^{T}_{1}.\mat{V}]\vec{r}}_{\hat{\vec{e}}}(mod\ q)
\end{aligned}
\end{equation}
If the norm $\|\hat{\vec{e}}\|< q/4$, then $|z|< q/4$ if and only if $m=0$, or else $m=1$. In this condition, our scheme is correct.  \qed
\subsection{Security}
In this part, we show the security proof of our IPE scheme as follows:
\begin{theorem}
Assuming the hardness of the standard LWE assumption, Our IPE scheme described above is weakly attribute hiding.
\end{theorem}
\noindent $Proof$: To prove the theorem we define a series games against adversary $\mathcal{A}$ that play the weak attribute hiding game. The adversary $\A$ outputs two attribute vectors $\vec{w}_{0}$ and $\vec{w}_{1}$ at the beginning of each game, and at some point output two messages $m_{0},m_{1}$. The first and last games correspond to real security game with challenge ciphertexts $\enc(\pp,\vec{w}_{0},m_{0})$ and $\enc(\pp,\vec{w}_{1},m_{1})$ respectively. 
In the intermediate games we use the "alternative" simulation algorithms $\si.\setup,\si.\keygen$, $\si.\enc$. During the course of the game the adversary can only request keys for predicate vector $\vec{v}$ such that $\langle \vec{v}, \vec{w}_{0} \rangle \neq 0$ and $\langle \vec{v}, \vec{w}_{1} \rangle \neq 0$.\\[0.2cm]
$\mat{Game\ 0}$: The challenger runs $\setup$, answers $\A$'s secret key queries using $\keygen$, and generates the challenge ciphertext using $\enc$ with attribute $\vec{w}_{0}$ and message $m_{0}$.\\[0.2cm]
$\mat{Game\ 1}$: The challenger runs $\si.\setup$ with $\vec{w}^{*}=\vec{w}_{0}$, and answers $\A$'s secret key queries using $\si.\keygen$. The challenger generates the challenge ciphertext using $\si.\enc$ with attribute  $\vec{w}_{0}$ and message $m_{0}$.\\[0.2cm]
$\mat{Game\ 2}$: The challenger runs $\si.\setup$ with $\vec{w}^{*}=\vec{w}_{0}$, and answers $\A$'s secret key queries using $\si.\keygen$. The challenger generates the challenge ciphertext by choosing a uniformly random element of the ciphertext space.\\[0.2cm]
$\mat{Game\ 3}$: The challenger runs $\si.\setup$ with $\vec{w}^{*}=\vec{w}_{1}$, and answers $\A$'s secret key queries using $\si.\keygen$. The challenger generates the challenge ciphertext by choosing a uniformly random element of the ciphertext space.\\[0.2cm]
$\mat{Game\ 4}$: The challenger runs $\si.\setup$ with $\vec{w}^{*}=\vec{w}_{1}$, and answers $\A$'s secret key queries using $\si.\keygen$. The challenger generates the challenge ciphertext using $\si.\enc$ with attribute  $\vec{w}_{1}$ and message $m_{1}$.\\[0.2cm]
$\mat{Game\ 5}$: The challenger runs $\setup$, answers $\A$'s secret key queries using $\keygen$, and generates the challenge ciphertext using $\enc$ with attribute $\vec{w}_{1}$ and message $m_{1}$.\\[0.2cm]
Now we define the "alternative" simulation algorithm as follows:\\[0.4cm]
$\si.\setup$ $(1^{\lambda},1^{l},\vec{w}^{*})$: On input a security parameter $\lambda$, a parameter $l$ denoting the length of predicate(and the attribute) vector, and an attribute vector $\vec{w}^{*} \in \Z^{l}_{q}$, do the following:
\begin{enumerate}
\item Choose a random matrix $\mat{A} \xleftarrow {\$} \Z^{n \times m}_{q}$ and a random vector $\vec{u} \xleftarrow {\$} \Z^{n}_{q}$.\

\item Choosing a random matrix $\mat{R}^{*} \xleftarrow {\$} \{-1,1\}^{m \times m}$.\

\item Define the matrix
\begin{equation}
\mat{W}^{'}= \begin{bmatrix}
w^{*}_{1} \mat{I}_{n}&w^{*}_{2} \mat{I}_{n}&\ldots&w^{*}_{l} \mat{I}_{n}
\end{bmatrix} \in \Z^{n \times ln}_{q},
\end{equation}
and set $\mat{B}=\mat{A}\mat{R}^{*}-\mat{W}^{'}.\G_{nl,l^{'},m} \in \Z^{n \times m}_{q}$.
\end{enumerate}
~~Output $\pp=(\mat{A}, \mat{B}, \vec{u}),\msk=(\mat{R}^{*}, \pp, \mat{T}_{\G_{n,2,m}})$.\\[0.4cm]
$\si.\keygen$ $(\pp, \msk, \vec{v})$: On input a master key $\msk$ and a vector $\vec{v} \in \Z^{l}_{q}$, do the following:
\begin{enumerate}
\item Check $\langle \vec{v}, \vec{w}^{*} \rangle =0$, if so ,output $\perp$.\

\item Define the matrix
\begin{equation}
\mat{V}^{'}= \begin{bmatrix}
v_{1} \mat{I}_{n}\\
v_{2} \mat{I}_{n}\\
\vdots\\
v_{l} \mat{I}_{n}
\end{bmatrix} \in \Z^{ln \times n}_{q},~~ \mat{V}=\G^{-1}_{nl,l^{'},m}(\mat{V}^{'}.\G_{n,2,m}).
\end{equation}
Set $\mat{U}=\mat{BV}, \mat{A}_{\vec{v}}=[\mat{A}|\mat{U}]=[\mat{A}|\mat{A}\mat{R}^{*}\mat{V}-\mat{W}^{'}\mat{V}^{'}\G_{n,2,m}]$.\

\item Let $\vec{r}\leftarrow$ $\sampleright$ $(\mat{A},\mat{R}^{*}\mat{V}, \mat{W}^{'}\mat{V}^{'}, \mat{T}_{\G_{n,2,m}}, \vec{u}, \sigma)$, so that $\mat{A}_{\vec{v}}.\vec{r}=\vec{u}$.
\end{enumerate}
~~Output the secret key $\sk_{\vec{v}}=\vec{r}$.\\[0.4cm]
$\si.\enc$ $(\pp,\vec{w},m,\msk)$: This algorithm is the same as the $\enc$ algorithm, except that $\mat{R}^{*}$ is used instead of the random matrix $\mat{R}$ in step4.\\[0.6cm]
We show that each pair of games($\mat{Game}$ $i$, $\mat{Game}$ $i+1$) are either statistically indistinguishable or computationally indistinguishable under the DLWE assumption.
\begin{lemma}
The view of the adversary $\A$ in $\mat{Game\ 0}$ is statistically closed to the view of $\A$ in $\mat{Game\ 1}$, similarly, $\mat{Game\ 4}$ is statistically closed to $\mat{Game\ 5}$.
\end{lemma}
\noindent $Proof$. We just prove the case for $\mat{Game\ 0}$ and $\mat{Game\ 1}$.\

First, The matrix $\mat{A}$ in $\pp$ is chosen by running $\trapgen$ in $\mat{Game\ 0}$, whereas it is a uniformly random matrix in $\Z^{n \times m}_{q}$ in $\mat{Game\ 1}$. Since $m$...................., by Theorem..................., the matrix $\mat{A}$ output by $\trapgen$  is statistically indistinguishable from a uniformly random matrix, and thus the distribution of $\mat{A}$ in $\mat{Game\ 0}$ and $\mat{Game\ 1}$ are statistically close.\

Next, We show that the distribution of \textbf{B} in $\pp$ and $\vec{c}_{1}$ in ciphertext in $\mat{Game\ 0}$ and $\mat{Game\ 1}$ are statistically indistinguishable. The difference between $(\mat{B},\vec{c}_{1})$ in two games is as follows:
\begin{itemize}
\item In $\mat{Game\ 0}$ the matrix $\mat{B}$ is uniformly random in $\Z^{n \times m}_{q}$. In $\mat{Game\ 1}$, $\mat{B}=\mat{A}\mat{R}^{*}-\mat{W}^{'}\G_{nl,l^{'},m}$. The matrix $\mat{R}^{*}$ is uniformly chosen from $\{-1,1\}^{m \times m}$.
\item In $\mat{Game\ 0}$ the challenge ciphertext components $\vec{c}_{1}$ are computed as $\vec{c}_{1}=\vec{s}^{T}(\mat{B}+\mat{W})+\vec{e}^{T}_{1}=\vec{s}^{T}(\mat{B}+\mat{W}^{'}\G_{nl,l^{'},m})+\vec{e}^{T}_{0}\mat{R}^{*}$, where $\mat{B}$ is uniformly random in $\Z^{n \times m}_{q}$ and the matrix $\mat{R}^{*}$ is uniformly chosen from $\{-1,1\}^{m \times m}$.\\
    In $\mat{Game\ 1}$ the challenge ciphertext components $\vec{c}_{1}$ is $\vec{c}_{1}=\vec{s}^{T}(\mat{A}\mat{R}^{*}-\mat{W}^{'}\G_{nl,l^{'},m}+\mat{W}^{'}\G_{nl,l^{'},m})+\vec{e}^{T}_{0}\mat{R}^{*}=\vec{s}^{T}\mat{A}\mat{R}^{*}+\vec{e}^{T}_{0}\mat{R}^{*}
    =(\vec{s}^{T}\mat{A}+\vec{e}^{T}_{0})\mat{R}^{*}$. Where $\mat{R}^{*}$ are the same matrix used to compute the public parameters $\mat{B}$. So the main difference between the two games is that the matrix $\mat{R}^{*}$ is chosen by $\enc$ and used only in the ciphertext $\vec{c}_{1}$ in $\mat{Game\ 0}$, whereas in $\mat{Game\ 1}$, it plays a double role: it's used to construct the matrix $\mat{B}$ in $\pp$ as well as the ciphertext $\vec{c}_{1}$.
\end{itemize}

Now we show that the distributions $(\mat{A}, \mat{B},\vec{c}_{1})$ in $\mat{Game\ 0}$ and $\mat{Game\ 1}$ are statistically indistinguishable. for $\mat{B}$ is uniformly random and $\mat{R}^{*}$ is uniformly random in $\{-1,1\}$, so by the $generalized$ leftover hash lemma, the following two distributions are statistically indistinguishable:
\begin{equation}
(\mat{A},\mat{A}\mat{R}^{*}-\mat{W}^{'}\G_{nl.l^{'},m}, \vec{e}^{T}_{0}\mat{R}^{*})\approx_{s}(\mat{A,B},\vec{e}^{T}_{0}\mat{R}^{*})
\end{equation}
We add the same quantity to both sides of Equation(14), we see that the following two distributions are statistically close:
\begin{equation}
\begin{split}
(\mat{A},\mat{A}\mat{R}^{*}-\mat{W}^{'}\G_{nl.l^{'},m},& \underbrace{\vec{s}^{T}(\mat{A}\mat{R}^{*}-\mat{W}^{'}\G_{nl.l^{'},m}+\mat{W}^{'}\G_{nl.l^{'},m})}
_{added\ term}+\vec{e}^{T}_{0}\mat{R}^{*})\\
&\approx_{s}(\mat{A,B},\underbrace{\vec{s}^{T}(\mat{B}+\mat{W}^{'}\G_{nl.l^{'},m})}_{added\ term}+\vec{e}^{T}_{0}\mat{R}^{*})\\
\end{split}
\end{equation}
The distribution on the left hand side of (15) is the distribution of $(\mat{A,B},\vec{c}_{1})$ in $\mat{Game\ 1}$, while the right hand side is the distribution in $\mat{Game\ 0}$. So the two distributions are statistically indistinguishable.\qed \\[0.2cm]

Next,we show the secret keys output by $\si.\keygen$ are statistically indistinguishable from those output by $\keygen$. Assuming $\sigma$ is sufficiently large, it follows from the properties of the algorithm $\sampleleft$ and $\sampleright$,  in both games the secret key is chosen from $\mathcal{D}_{\Lambda^{\vec{u}}_{q}(\mat{A}_{v}),\sigma}$ with overwhelming probability, so the two keys are statistically indistinguishable.
\begin{lemma}
If the DLWE assumption holds, then the view of $\A$ in $\mat{Game\ 1}$ is computationally indistinguishable from the view of $\A$ in $\mat{Game\ 2}$, similarly, $\mat{Game\ 3}$ is computationally indistinguishable from $\mat{Game\ 4}$.
\end{lemma}
\noindent $Proof$. We just prove the case for $\mat{Game\ 1}$ and $\mat{Game\ 2}$.\

Suppose we are given $m+1$ LWE instances $(\vec{a}_{i},b_{i})$ for $i=0,...,m$, where either $b_{i}=\vec{s}^{T}\vec{a}_{i}+e_{i}$ for some fixed random secret $\vec{s}\xleftarrow{\$} \Z^{n}_{q}$ and  Gaussian noise $e_{i}\leftarrow D_{\Z_{q},\alpha}$ or $b_{i}$ is uniformly random in $\Z_{q}$. Define the following variables:
\begin{equation}
\begin{split}
&\mat{A}=\begin{bmatrix}
\vec{a}_{1}&\ldots&\vec{a}_{m}
\end{bmatrix} \in \Z^{n \times m}_{q} ~~~~~~~ \vec{u}=\vec{a}_{0}\\
&\vec{c}_{0}=(b_{1},...,b_{m}) \in \Z^{m}_{q}~~~~~~~~~~~~c_{2}=b_{0}+\lfloor \frac{q}{2}\rfloor m\\
\end{split}
\end{equation}
We simulate the challenger as follows:
\begin{itemize}
\item $\mat{Setup}$: Run $\si.\setup$ with $\vec{w}^{*}=\vec{w}_{0}$. and let $\mat{A}$, $\vec{u}$ as defined above.

\item $\mat{Secret key queries}$: Run $\si.\keygen$ .

\item $\mat{Challenge ciphertext}$: Let $\vec{c}_{1}=\vec{c}_{0}\mat{R}^{*}$(using the $\mat{R}^{*} \in \msk$). Output $(\vec{c}_{0}, \vec{c}_{1}, c_{2})$.
\end{itemize}
In the $\si.\enc$ algorithm it sets:
\begin{equation}
\begin{split}
\vec{c}_{1}&=\vec{s}^{T}(\mat{A}\mat{R}^{*}-\mat{W}^{'}\G_{nl,l^{'},m}+\mat{W}^{'}\G_{nl,l^{'},m})+\vec{e}^{T}_{0}\mat{R}^{*}=\vec{s}^{T}\mat{A}\mat{R}^{*}+\vec{e}^{T}_{0}\mat{R}^{*}\\
&=(\vec{s}^{T}\mat{A}+\vec{e}^{T}_{0})\mat{R}^{*}.
\end{split}
\end{equation}
So if $b_{i}=\vec{s}^{T}\vec{a}_{i}+e_{i}$, then $\vec{c}_{1}=\vec{c}_{0}\mat{R}^{*}$ and the simulator is the same as a $\mat{Game\ 1}$ challenger. Whereas if $b_{i}$ is random in $\Z_{q}$, then simulated ciphertext is $(\vec{c}_{0}, \vec{c}_{0}\mat{R}^{*}, c_{2})$. By the leftover hash lemma, $\vec{c}_{0}\mat{R}^{*}$ is uniformly random. Thus the ciphertext in this case is uniformly random and the simulator is identical to $\mat{Game\ 2}$'s challenger.\

We conclude that any efficient adversary that can distinguish $\mat{Game\ 1}$ from $\mat{Game\ 2}$ can solve the DLWE problem.\qed
\begin{lemma}
The view of $\A$ in $\mat{Game\ 2}$ is statistically indistinguishable from the view of $\A$ in $\mat{Game\ 3}$.
\end{lemma}
\noindent $Proof$. Note that the only difference between the two games is that the attribute vector $\vec{w}^{*}$ is equal to $\vec{w}_{0}$ in $\mat{Game\ 2}$, whereas $\vec{w}^{*}=\vec{w}_{0}$ in $\mat{Game\ 3}$. The attribute vector $\vec{w}^{*}$ appears in the public parameter $\mat{B}=\mat{A}\mat{R}^{*}-\mat{W}^{'}\G_{nl,l^{'},m}$, by the $generalized$ leftover hash lemma $(\mat{A},\mat{AR}^{*})$ is statistically indistinguishable from ($\mat{A}$,$\mat{U}_{uni}$), where $\mat{U}_{uni}$ is uniformly random. So $\mat{A}\mat{R}^{*}-\mat{X}$ for any fixed $\mat{X}$ is also uniformly random. 
It follows that the distribution of $\mat{B}$ in the two games are statistically close. \qed \\[0.3cm]
Now we conclude the proof of the Theorem. Suppose that there is an efficient adversary $\A$ that can win the security game. Let $\A^{(i)}$ denote the output of $\A$ in $\mat{Game}$ $i$. We have
\begin{equation}
|Pr[\A^{(0)}=1]-Pr[\A^{(5)}=1]|\geq\frac{1}{poly(n)}.
\end{equation}
By a standard hybrid argument, this implies that
\begin{equation}
|Pr[\A^{(i)}=1]-Pr[\A^{(i+1)}=1]|\geq\frac{1}{poly(n)}.
\end{equation}
for $i=0,...,4$. Since $\A$ is polynomial time, $\mat{Lemma(4.3)}$ implies that (19) cannot hold for $i=0$ or 4, $\mat{Lemma(4.5)}$ implies that (19) cannot hold for $i=2$. So from $\mat{Lemma(4.4)}$ $\A$ can be used to solve the DLWE problem. \qed


\subsection{Parameter Setting}
We set the system parameters as follows, in which the $\delta>0$ is an arbitrarily small constant:
\begin{table}[htbp]
\centering
\begin{tabular}{|c|c|c|}
\hline parameters & Description & Setting\\
\hline $\lambda$ & security parameter & -\\
\hline $n$ & $\pp$-lattice row dimension & $n=\lambda$\\
\hline $m$ & $\pp$-lattice column dimension & $m=n^{1+\delta}$\\
\hline $q$ & modulus & $q=n^{4.5+4\delta}\log^{3.5+2\delta}(n)$\\
\hline $s$ & $\sampleleft$ and $\sampleright$ width & $s=n^{1.5+1.5\delta}\log^{1.5+\delta}(n)$\\
\hline $\alpha$ & error width & $\alpha=\sqrt{n}\log^{1+\delta}(n)$\\
\hline $l$ & predicate vectors and attribute vectors dimension & $l=\log\log(n)$\\
\hline $l^{'}$ & integer-base parameter & $l^{'}=\log(n)$\\
\hline
\end{tabular}
\caption{Parameters and Example setting}
\end{table}\\[0.4cm]
We chose the parameters to satisfy the following constraints:
\begin{itemize}
 \item To ensure correctness, it requires that $|e-[\vec{e}^{T}_{0}|\vec{e}^{T}_{1}.\mat{V}]\vec{r}|< q/4$; we just need to bound the dominating term. So we parse the  vector $\vec{r}$ into two equal-length vectors $\vec{r}_{1}$ and $\vec{r}_{2}$, then:
 \begin{equation}
       |\vec{e}^{T}_{1}.\mat{V}.\vec{r}_{2}|\leq \| \vec{e}^{T}_{0}\|.\| \mat{R}\|.\| \mat{V}\|.\| \vec{r}_{2}\|\approx \alpha\sqrt{m}.\sqrt{m}.l^{'}m.s\sqrt{m}=m^{2.5}s\alpha l^{'}< q/4
\end{equation}
 \item For $\sampleleft$, we know $\| \widetilde{\mat{T}_{\mat{A}}}\|=O(\sqrt{n\log(q)})$, so it needs that the Gaussian parameter $s$ satisfies
 \begin{equation}
 s>\sqrt{n\log(q)}.\omega(\sqrt{\log(m)}).
 \end{equation}
 \item For $\sampleright$, it's known that $\| \widetilde{\mat{T}_{\mat{A}}}\|\leq 5$ and in the proof of security
 \begin{equation}
 s_{\mat{R}}\leq \| \mat{R}^{*}\|.\| \mat{V}\|\leq 12\sqrt{2m}.(l^{'}m)=O(2^{l}m^{1.5})
 \end{equation}
 Therefore, we need the Gaussian parameter $s$ to satisfy a stronger constraint
 \begin{equation}
 s>2^{l}m^{1.5}\omega(\sqrt{\log(m)}).
 \end{equation}
 \item To apply the Leftover Hash Lemma, we need $m\geq (n+1)\log(q)+\omega(\log(n))$.
 \item For the hardness of DLWE assumption, we apply Regev's reduction, then we need $\alpha >\sqrt{n}\omega(\log(n))$.
\end{itemize}
\textbf{Remark}. Regev [Reg.....] showed that there exists an efficiently samplable $B$-bounded distribution $\chi$ for $B\geq \sqrt{n}.\omega(\log(n))$, so that if there is an efficient algorithm that solves the (average-case)$\LWE_{n,m,q,\chi}$ problem, then there is an efficient quantum algorithm that solves $\Gapsvp_{\tilde{O}(n.q/B)}$ and $\Sivp_{\tilde{O}(n.q/B)}$ on any $n$-dimensional lattice. The work of Brakerski et al.[BLP......] shows an analogous-but-classical result for any $\sqrt{n}$-dimensional lattice. 
For the case of our parameter setting ,the resulting lattice problems' approximation factor is $\widetilde{O}(n^{5+\epsilon})$, for $\epsilon>0$.\\[0.6cm]
\textbf{Further Optimizations}. We just use a trivial bound $l^{'}m$ to bound the matrix $\textbf{V}$. However, the $\textbf{V}$ is highly sparse in fact, it's possible to optimize $\textbf{V}$; e.g. about $l^{'}\log_{2}(q)$. Then the parameters $s$ and $q$ can be selected more aggressively, e.g. $s\approx \sqrt{m}$ and $q\approx n^{2.5}$, this will result in a better approximation factor of about $\widetilde{O}(n^{3+\epsilon})$.\

Using the tag-based sampling algorithms of [MP12....] in place of [ABB.....]'s $\sampleright$ procedure leads to an approximation factor of $\widetilde{O}(n^{1.5})$, which matches the approximation factor of Dual Regev type $\mat{PKE}$, up to polylogarithmic factors in the security parameter $n$.



\section{Conclusion and Open Problems} 

\newpage
\bibliographystyle{alpha}
\bibliography{abbrev3,crypto_crossref,extra}

\begin{appendix}
\section{Hierarchical Inner-Product Encryption}
We use the definition proposed by Okamoto and Takashima [OT09.....]. We call a tuple of positive integers $\overrightarrow{\mu}:=(n,d;\mu_{1},...,\mu_{d})$ s.t $\mu_{0}=0<\mu_{1}<\mu_{2}<...<\mu_{d}=n$ a format of hierarchy of depth $d$ attribute spaces. Let $\sum_{l}(l=1,..,d)$ be the sets of attributes, where each $\sum_{l}:= \F^{\mu_{l}-\mu_{l-1}}_{q} \backslash \{\overrightarrow{0}\}$. Let the hierarchical attributes $\sum:=\bigcup^{d}_{l=1}(\sum_{1} \times ...\times \sum_{l})$, where the union is a disjoint union. Then, for $\overrightarrow{v}_{i} \in \F^{\mu_{l}-\mu_{l-1}}_{q} \backslash \{\overrightarrow{0}\}$, the hierarchical predicate $f_{(\overrightarrow{v}_{1},...,\overrightarrow{v}_{l})}(\overrightarrow{x}_{1},...,\overrightarrow{x}_{h})=1$ iff $l\leq h$ and $\overrightarrow{x}_{i}.\overrightarrow{v}_{i}=0$ for all $i$ s.t. $a\leq i \leq l$.\

Let the space of hierarchical predicates $\mathcal{F}=:\{f_{(\overrightarrow{v}_{1},...,\overrightarrow{v}_{l})}|\overrightarrow{v}_{i} \in \F^{\mu_{l}-\mu_{l-1}}_{q}\backslash \{\overrightarrow{0}\}\}$. We call $h$ (resp.$l$) the level of $(\overrightarrow{x}_{1},...,\overrightarrow{x}_{h})$ (resp.$(\overrightarrow{v}_{1},...,\overrightarrow{v}_{h})$.
\begin{definition}
Let $\overrightarrow{\mu}:=(n,d;\mu_{1},...,\mu_{d})$ s.t $\mu_{0}=0<\mu_{1}<\mu_{2}<...<\mu_{d}=n$ be a format of hierarchy of depth $d$ attribute spaces. A hierarchical predicate encryption (HPE) scheme for the class of hierarchical inner-product predicates $\mathcal{F}$ over the set of hierarchical attributes $\sum$ consists of probabilistic polynomial time algorithms $\setup, \keygen, \enc, \dec$ and $\delegate_{l}$ for $l=1,...,d-1$. They are given as follows:
\begin{itemize}
\item $\setup$ takes as input security parameter $1^{\lambda}$ and format of hierarchy $\overrightarrow{\mu}$, and outputs master public key $\pp$ and master secret key $\msk$.
\item $\keygen$ takes as input the $\pp, \msk$, and predicate vectors $(\overrightarrow{v}_{1},...,\overrightarrow{v}_{l})$. It outputs a corresponding secret key $\sk_{(\overrightarrow{v}_{1},...,\overrightarrow{v}_{l})}$.
\item $\enc$ takes as input the $\pp$, attribute vectors $(\overrightarrow{x}_{1},...,\overrightarrow{x}_{h})$, where $1\leq h \leq d$, and plaintext $m$ in some associated plaintext space, $\mat{msg}$. It returns ciphertext $c$.
\item $\dec$ takes as input the master public key $\pp$, secret key $\sk_{(\overrightarrow{v}_{1},...,\overrightarrow{v}_{l})}$, where $1\leq l \leq d$, and ciphertext $c$. It outputs either plaintext $m$ or the distinguished symbol $\perp$.
\item $\delegate_{l}$ takes as input the master public key $\pp$, $l$-th level secret key $\sk_{(\overrightarrow{v}_{1},...,\overrightarrow{v}_{l})}$, and $(l+1)$-th level predicate vector $\overrightarrow{v}_{l+1}$. It returns $(l+1)$-th level secret key $\sk_{(\overrightarrow{v}_{1},...,\overrightarrow{v}_{l+1})}$.
\end{itemize}
\end{definition}
\paragraph{Correctness.} For all correctly generated $\pp$ and $\sk_{(\overrightarrow{v}_{1},...,\overrightarrow{v}_{l})}$, generate $c \xleftarrow{R} \enc(\pp, m, (\overrightarrow{x}_{1},...,\overrightarrow{x}_{h}))$ and $m^{'}= \dec(\pp, \sk_{(\overrightarrow{v}_{1},...,\overrightarrow{v}_{l})}, c)$. If $f_{(\overrightarrow{v}_{1},...,\overrightarrow{v}_{l})}(\overrightarrow{x}_{1},...,\overrightarrow{x}_{h})=1$, then $m^{'}=m$. Otherwise, $m^{'}\neq m$ except for negligible probability. For $f$ and $f^{'}$ in $\mathcal{F}$, we denote $f^{'}\leq f$ if the predicate vector for $f$ is the prefix of that for $f^{'}$.

\paragraph{Security.} A hierarchical inner-product predicate encryption scheme for hierarchical predicates $\mathcal{F}$ over hierarchical attributes $\sum$ is \textbf{selectively attribute-hiding against plaintext attacks} if for all probabilistic polynomial-time adversaries $\A$, the advantage of $\A$ in the following experiment is negligible in the security parameter.
\begin{enumerate}
\item $\A$ outputs challenge attribute vectors $X^{(0)}=(\overrightarrow{x}^{(0)}_{1},....,\overrightarrow{x}^{(0)}_{h}), X^{(1)}=(\overrightarrow{x}^{(1)}_{1},....,\overrightarrow{x}^{(1)}_{h})$.
\item $\setup$ is run to generate keys $\pp$ and $\msk$, and $\pp$ is given to $\A$.
\item $\A$ may adaptively makes a polynomial number of queries of the following type:
 \begin{itemize}
 \item $\A$ asks the challenger to create a secret key for a predicate $f \in \mathcal{F}$. The challenger creates a key for $f$ without giving it to $\A$.
 \item $\A$ specifies a key for predicate $f$ that has already been created, and asks the challenger to perform a delegation operation to create a child key for $f^{'} \leq f$. The challenger computes the child key without giving it to the adversary.
 \item $\A$ asks the challenger to reveal an already-created key for predicate $f$ s.t. $f(X^{(0)})=f(X^{(1)})=0$.
 \end{itemize}
 Note that when key creation requests are made, $\A$ does not automatically see the created key. $\A$ see a key only when it makes a reveal key query.
\item $\A$ outputs challenge plaintexts $m_{0},m_{1}$.
\item A random bit $b$ is chosen. $\A$ is given $c\xleftarrow{R} \enc(\pp, m_{b}, X^{(b)})$.
\item The adversary may continue to request keys for additional predicate vectors subject to the restrictions given in step 3.
\item $\A$ outputs a bit $b^{'}$, and succeeds if $b^{'}=b$.
\end{enumerate}
We define the advantage of $\A$ as the quantity $\advantage^{HIPE}_{\A}(\lambda)=|Pr[b^{'}=b]-1/2|$.
\subsection{Our HIPE Scheme}
We can apply the technical in our basic IPE scheme to optimize the hierarchical inner-product encryption scheme proposed by [Xagawa......].  For $i \in [d]$, we denote $s_{i}$ to be the Gaussian parameter used in the $\keygen$ and $\delegate$ algorithm. Roughly speaking, we stack matrices $\mat{B}_{\vec{v}_{i}}$ to make IPE hierarchical.\\[0.4cm]
$\setup(1^{\lambda}, n, q, m, \vec{\mu})$. On input a security parameter $\lambda$, and an hierarchical format of depth $d$ $\vec{\mu}=(l,d;\mu_{1},...,\mu_{d})$:
\begin{enumerate}
\item $(\mat{A}, \mat{T}_\mat{A})\leftarrow \trapgen(1^{\lambda}, q, n, m)$.
\item Chose a random matrix $\mat{B}\xleftarrow{\$} \Z^{n \times m}_{q}$ and a random vector $\vec{u}\xleftarrow{\$} \Z^{n}_{q}$.
\end{enumerate}
~~~~~Output $\pp=(\mat{A}, \mat{B}, \vec{u})$ and $\msk= \mat{T}_\mat{A}$.\\[0.4cm]
$\keygen(\pp, \msk, \overrightarrow{\mat{V}})$: On input $\pp, \msk$ and predicate vectors $\overrightarrow{\mat{V}}=(\vec{v}_{1},...,\vec{v}_{j})$ where $\vec{v}_{i}=(v_{i,1},...,v_{i,l})$:
\begin{enumerate}
\item For all $i \in [j]$, define the matrix
\begin{equation}
\mat{V}^{'}_{i}= \begin{bmatrix}
v_{i,1} \mat{I}_{n}\\
v_{i,2} \mat{I}_{n}\\
\vdots\\
v_{i,l} \mat{I}_{n}
\end{bmatrix} \in \Z^{ln \times n}_{q},~~~ \mat{V}_{i}=\G^{-1}_{nl,l^{'},m}(\mat{V}^{'}_{i}.\G_{n,2,m}) \in [l^{'}]^{m \times m}.
\end{equation}
\item Set  $\mat{A}_{\overrightarrow{\mat{V}}}=[\mat{A} | \mat{B}\mat{V}_{1}| ...... | \mat{B}\mat{V}_{j}] \in \Z^{n \times (j+1)m}_{q}$.
\item Using the master secret key $\mat{T}_\mat{A}$ to construct short basis $\mat{E}_{\overrightarrow{\mat{V}}}$ for  $\Lambda^{\perp}_{q}(\mat{A}_{\overrightarrow{\mat{V}}})$  by  invoking the $\sampleleft$ algorithm.
\end{enumerate}
~~~~~Output $\sk_{\overrightarrow{\mat{V}}}=\mat{E}_{\overrightarrow{\mat{V}}}$.\\[0.4cm]
$\enc(\pp, \overrightarrow{\mat{W}}, m)$: On input $\pp$, a message $m \in \{0,1\}$, and attribute vectors $\overrightarrow{\mat{W}}=(\vec{w}_{1},...,\vec{w}_{h})$ where $\vec{w}_{i}=(w_{i,1},...,w_{i,l})$:
\begin{enumerate}
\item Choose a random vector $\vec{s} \xleftarrow{\$} \Z^{n}_{q}$.
\item Set $\vec{c}_{0}\leftarrow \vec{s}^{T}\textbf{A}+ \vec{e}^{T}_{0}$, where $\vec{e}_{0}\leftarrow \chi^{m}$.
\item Set $c\leftarrow \vec{s}^{T}\vec{u}+e+\lfloor q/2 \rfloor m$, where $e\leftarrow \chi$.
\item For all $i \in [h]$, define the matrix
\begin{equation}
\mat{W}^{'}_{i}= \begin{bmatrix}
w_{i,1} \mat{I}_{n}&w_{i,2} \mat{I}_{n}&\ldots&w_{i,l} \mat{I}_{n}
\end{bmatrix} \in \Z^{n \times ln}_{q},~~~ \mat{W}_{i}=\mat{W}^{'}_{i}.\G_{nl,l^{'},m} \in \Z^{n \times m}_{q}.
\end{equation}
choose a random matrix $\mat{R}_{i} \xleftarrow{\$} \{-1,1\}^{m \times m}$ and set $\vec{c}_{i} \leftarrow \vec{s}^{T}(\mat{B}+ \mat{W}^{'}_{i})+ \vec{e}^{T}_{0}\mat{R}_{i}$.
\end{enumerate}
~~~~~Output $\ct=(\vec{c}_{0}, \{\vec{c}_{i}\}, c)$.\\[0.4cm]
$\dec(\pp, \sk_{\overrightarrow{\mat{V}}}, \ct)$: On input $\pp$, a decryption key $\sk_{\overrightarrow{\mat{V}}}$ where $\overrightarrow{\mat{V}}=(\vec{v}_{1},...,\vec{v}_{j})$, and a ciphertext $\ct=(\vec{c}_{0}, \{\vec{c}_{i}\}, c)$:
\begin{enumerate}
\item For $i \in [j]$, set $\vec{c}_{\vec{v}_{i}}=\vec{c}_{i}.\mat{V}_{i}$.
\item Let $\vec{c} \leftarrow [\vec{c}_{0},\vec{c}_{\vec{v}_{1}}...,\vec{c}_{\vec{v}_{j}}] \in \Z^{(j+1)m}_{q}$.
\item Set $\tau_{t}=\sigma_{t}.\sqrt{(t+1)m}.\omega(\sqrt{(t+1)m})$, Then $\tau_{t}\geq \| \widetilde{\mat{E}_{\overrightarrow{\mat{V}}}} \|.\omega(\sqrt{(t+1)m})$.
\item Compute $\vec{x}_{\overrightarrow{\mat{V}}} \leftarrow \samplepre(\mat{A}_{\overrightarrow{\mat{V}}}, \mat{E}_{\overrightarrow{\mat{V}}}, \vec{u}, \tau_{t})$.
\item Compute $z=c-\vec{c}.\vec{x}_{\overrightarrow{\mat{V}}}$ (mod $q$), and output $\lfloor (2/q)z \rceil \in \{0,1\}$.
\end{enumerate}
$\delegate(\pp, \sk_{\overrightarrow{\mat{V}}}, \mat{V}^{'})$: On input $\pp$, a decryption key $\sk_{\overrightarrow{\mat{V}}}=\mat{E}_{\overrightarrow{\mat{V}}}$, where $\overrightarrow{\mat{V}}=(\vec{v}_{1},...,\vec{v}_{j})$, and $\overrightarrow{\mat{V}}^{'}=(\vec{v}_{1},...,\vec{v}_{j},\vec{v}_{j+1},...,\vec{v}_{t})$, $\vec{v}_{i}=(v_{i,1},...,v_{i,l})$. do:
\begin{enumerate}
\item For all $i \in [t]$, define the matrices
\begin{equation}
 \mat{V}^{'}_{i}= \begin{bmatrix}
v_{i,1} \mat{I}_{n}\\
v_{i,2} \mat{I}_{n}\\
\vdots\\
v_{i,l} \mat{I}_{n}
\end{bmatrix} \in \Z^{ln \times n}_{q},~~~ \mat{V}_{i}=\G^{-1}_{nl,l^{'},m}(\mat{V}^{'}_{i}.\G_{n,2,m}) \in [l^{'}]^{m \times m}.
\end{equation}
\item Set $\mat{A}_{\overrightarrow{\mat{V}}^{'}}=[\mat{A} | \mat{B}\mat{V}_{1}| ...... | \mat{B}\mat{V}_{t}] \in \Z^{n \times (t+1)m}_{q}$
\item Recall that the secret key $\mat{E}_{\overrightarrow{\mat{V}}}$ is a short basis for $\Lambda^{\perp}_{q}(\mat{A}_{\overrightarrow{\mat{V}}})$. Using it to construct a short basis for $\Lambda^{\perp}_{q}(\mat{A}_{\overrightarrow{\mat{V}}^{'}})$ by invoking
\begin{equation}
\mat{E}_{\overrightarrow{\mat{V}}^{'}} \leftarrow \sampleleft ( \mat{A}_{\overrightarrow{\mat{V}}}, [\mat{B}\mat{V}_{j+1}|...|\mat{B}\mat{V}_{t}], \sk_{\overrightarrow{\mat{V}}}, \sigma_{t}).
\end{equation}
\end{enumerate}
~~~~~Output $\sk_{\overrightarrow{\mat{V}}^{'}}=\mat{E}_{\overrightarrow{\mat{V}}^{'}}$
\paragraph{Correctness.} It's easy to verify that $\vec{c}_{\vec{v}_{i}}=\vec{s}^{T}\mat{B}\mat{V}_{i}+\vec{e}^{T}_{0}\mat{R}_{i}\mat{V}_{i}$, so $\vec{c}=\vec{s}^{T}\mat{A}_{\overrightarrow{\mat{V}}}+[\vec{e}^{T}_{0}|\vec{e}^{T}_{0}\mat{R}_{1}\mat{V}_{1}|...|\vec{e}^{T}_{0}\mat{R}_{i}\mat{V}_{i}]$. Then $z=c-\vec{c}.\vec{x}_{\overrightarrow{\mat{V}}}=\lfloor q/2 \rfloor m+\underbrace{e-[\vec{e}^{T}_{0}|\vec{e}^{T}_{0}\mat{R}_{1}\mat{V}_{1}|...|\vec{e}^{T}_{0}\mat{R}_{i}\mat{V}_{i}]\vec{x}_{\overrightarrow{\mat{V}}}}_{\vec{e}}(mod\ q)$. If the norm of the error term $\vec{e}$ is small, our HIPE scheme is correct.

\paragraph{Security.} We omit the security proof of our HIPE scheme, since it's very similar to our basic IPE scheme and [ADCM......]. We also omit the parameters setting.


\section{Fuzzy Identity-based Encryption}
In this section, we construct a FIBE scheme from our IPE scheme, we first introduce the definition and the security model of Fuzzy IBE.\\[0.4cm]
A Fuzzy Identity Based encryption scheme consists of the following four algorithms:
\begin{description}
 \item $\fuzzy.\setup(\lambda, l)\rightarrow (\pp, \msk)$: The algorithm takes as input the security parameter $\lambda$ and the maximum length of identities $l$. It outputs the public parameters $\pp$, and the master secret key $\msk$.
 \item $\fuzzy.\extract(\msk, \pp, id, k)$: This algorithm takes as input the master key $\msk$, the public parameters $\pp$, an identity $id$ and the threshold  $k\leq l$. It outputs a decryption key $\sk_{id}$.
 \item $\fuzzy.\enc(\pp, m, id^{'}) \rightarrow \ct_{id^{'}}$: The algorithm takes as input: a message bit $m$, an identity $id^{'}$, and the public parameters $\pp$. It outputs the ciphertext $\ct_{id^{'}}$.
 \item $\fuzzy.\dec(\pp, \ct_{id^{'}}, \sk_{id})\rightarrow m$: This algorithm takes as input the ciphertext $\ct_{id^{'}}$, the decryption key $\sk_{id}$ and the public parameters $\pp$. It outputs the message $m$ if $|id \bigcap id^{'}| \geq k$.
\end{description}
\paragraph{Security.} We follow the Selective-ID model of security of Fuzzy Identity Based Encryption as given by Sahai and Waters[SW......].
\begin{itemize}
 \item \textbf{Target}: The adversary declares the challenge identity, $id^{*}$, and he wishes to be challenged upon.
 \item \textbf{Setup}: The challenger runs the Setup algorithm of Fuzzy-IBE and gives the public parameters to the adversary.
 \item \textbf{Phase 1}: The adversary is allowed to issue queries for private keys for identities $id_{j}$ of its choice, as long as $|id_{j} \bigcap id^{*}|< k; \forall j$
 \item \textbf{Challenge}: The adversary submits a message to encrypt. the challenger encrypts the message with the challenge $id^{*}$ and then flips a random coin $r$. If $r=1$, the ciphertext is given to the adversary, otherwise a random element of the ciphertext space is returned.
 \item \textbf{Phase 2}: Phase 1 is repeated.
 \item \textbf{Guess}: the adversary outputs a guess $r^{'}$ of $r$. the advantage of an adversary A in this game is defined as $|Pr[r^{'}=r]-1/2|$.
\end{itemize}

A Fuzzy Identity Based Encryption scheme is secure in the Selective-Set model of security if all polynomial time adversaries have at most a negligible advantage in the Selective-Set game.\\[0.4cm]
We introduce the embedding of exact threshold by Katz, Sahai, and Waters[KSW......].
\paragraph{Exact threshold}: For binary vector $\vec{x} \in \{0,1\}^{N}$, $H_{w}(\vec{x}$ denotes the Hamming weight of $\vec{x}$. For binary vectors $\vec{a}, \vec{x} \in \{0,1\}^{n}$, the exact threshold predicate is denoted by $\mathcal{P}^{th}_{=t}(\vec{a}, \vec{x})$ and output 1 if and only if $H_{w}(\vec{a} \& \vec{x})=t$, where $\&$ denotes the logical conjunction. Suppose that $t<q$. Set $\mu=N+1$, $\vec{v}=(\vec{a}, 1) \in \Z^{\mu}_{q}$, and $\vec{w}=(\vec{x}, -t) \in \Z^{\mu}_{q}$. We have that $\langle \vec{v}, \vec{w} \rangle =0 $ if and only if $H_{w}(\vec{a} \& \vec{x})=t$.

\subsection{Our FIBE scheme}
Now, we use our basic IPE scheme to construct a FIBE scheme. Let $\{0,1\}^{N}$ be a space of identities. The threshold predicate over $\{0,1\}^{N}$ is defined by $\mathcal{P}^{th}_{\geq t}(\vec{a}, \vec{x})$ and output 1 if and only if $H_{w}(\vec{a} \& \vec{x}) \geq t$.\

It's easy to see that the above predicate can be written as $\bigcup^{N}_{i=t}\mathcal{P}^{th}_{i=t}(\vec{a}, \vec{x})$. Hence, we can implement a FIBE scheme in a lazy way by repeating ciphertexts of an IPE scheme that supports the relations $\mathcal{P}^{th}_{\geq t}$ for $i=t,...,N$.\\[0.4cm]
$\fuzzy.\setup(1^{\lambda}, 1^{N})$: On input a security parameter $\lambda$, and identity size $N$, do:
\begin{enumerate}
\item $(\textbf{A}, \mat{T}_\mat{A})\leftarrow \trapgen(1^{\lambda}, q, n, m)$.
\item Chose a random matrix $\mat{B}\xleftarrow{\$} \Z^{n \times m}_{q}$ and a random vector $\vec{u}\xleftarrow{\$} \Z^{n}_{q}$.
\end{enumerate}
~~~~~Output $\pp=(\textbf{A}, \textbf{B}, \vec{u})$ and $\msk= \mat{T}_\mat{A}$.\\[0.4cm]
$\fuzzy.\extract(\pp, \msk, id, t)$: On input public parameters $\pp$, a master key $\msk$, an identity $id= (a_{1},...,a_{N}) \in \{0,1\}^{N}$ and threshold $t \leq N$, do:
\begin{enumerate}
\item Set vector $\vec{v}=(a_{1},...,a_{N},1) \in \Z^{\mu}_{q}$.
\item Define the matrix
\begin{equation}
 \mat{V}^{'}= \begin{bmatrix}
a_{1} \mat{I}_{n}\\
a_{2} \mat{I}_{n}\\
\vdots\\
a_{N} \mat{I}_{n}\\
\mat{I}_{n}
\end{bmatrix} \in \Z^{\mu n \times n}_{q},~~~ \mat{V}=\G^{-1}_{\mu n,l^{'},m}(\mat{V}^{'}_{i}.\G_{n,2,m}) \in [l^{'}]^{m \times m}.
\end{equation}
\item Define the  matrix $\mat{U}=\mat{BV} \in \Z^{n \times m}_{q}$, $\mat{A}_{id}=[\mat{A}|\mat{U}]$.
\item Using the master secret key $\msk=(\mat{T}_\mat{A},\sigma)$, compute $\vec{r}\leftarrow$ $\sampleleft$ ($\mat{A}, \mat{T}_\mat{A},\mat{U}, \vec{u},\sigma$). Then $\vec{r}$ is a vector in $\Z^{2m}$ satisfying $\mat{A}_{id}.\vec{r}=\vec{u}$ (mod $q$).
\end{enumerate}
~~Output the secret key $\sk_{id}=\vec{r}$.\\[0.4cm]
$\fuzzy.\enc(\pp, id^{'}, m)$: On input public parameters $\pp$, an identity $id^{'}=(x_{1},...,x_{N}) \in \{0,1\}^{N}$, and a message $m \in \{0,1\}$, do:
\begin{enumerate}
\item Define a sequence of vectors $\vec{w}_{i}=(x_{1},...,x_{N},-t-i+1) \in \Z^{\mu n \times n}_{q}, i=1,...,N-t+1$.
\item Choose a uniformly random $\vec{s}\xleftarrow{\$} \Z^{n}_{q}$.
\item Choose a noise vector $\vec{e}_{0}\leftarrow D_{\Z^{m}_{q},\alpha}$ and a noise term $e\leftarrow D_{\Z_{q},\alpha}$.
\item Compute $\vec{c}_{0}=\vec{s}^{T}\mat{A}+\vec{e}^{T}_{0}$.
\item For all $i \in [N-t+1]$ Define the matrix
\begin{equation}
\mat{W}^{'}_{i}= \begin{bmatrix}
x_{1} \mat{I}_{n}&x_{2} \mat{I}_{n}&\ldots&x_{N} \mat{I}_{n}& -(t+i-1)\mat{I}_{n}
\end{bmatrix} \in \Z^{n \times \mu n}_{q},~~~ \mat{W}_{i}=\mat{W}^{'}_{i}.\G_{\mu n,l^{'},m} \in \Z^{n \times m}_{q}.
\end{equation}
Pick a sequence of random matrices $\mat{R}_{i}\xleftarrow{\$} \{-1,1\}^{m \times m}, i=1,...,N-t+1$, define error vectors $\vec{e}^{T}_{i}=\vec{e}^{T}_{0}\mat{R}_{i}$. Set
\begin{equation}
\vec{c}_{i}=\vec{s}^{T}(\mat{B}+\mat{W}_{i})+\vec{e}^{T}_{i},~~~c=\vec{s}^{T}\vec{u}+e+\lfloor \frac{q}{2} \rfloor m.
\end{equation}
\end{enumerate}
~~Output ciphertext  $\ct_{id^{'}}=(\vec{c}_{0},\{\vec{c}_{i}\},c)$.\\[0.4cm]
$\fuzzy.\dec(\pp, \sk_{id}, \ct_{id^{'}})$: On input public parameters $\pp$, a decryption key $\sk_{id}$, and a ciphertext $\ct_{id^{'}}$, do:
\begin{enumerate}
\item Let $H_{w}(id \& id^{'})$ denotes Hamming weight of the logical conjunction of $id$ and $id^{'}$. If $H_{w}(id \& id^{'})=k<t$, output $\perp$. Otherwise, we parse the $\ct_{id^{'}}=(\vec{c}_{0},\{\vec{c}_{i}\},c)$, and compute the matrix $\mat{V}$ for $id$ as above.
\item Compute $\tilde{\vec{c}}_{k-t+1}=\vec{c}_{k-t+1}.\mat{V}$, let $\vec{c}=[\vec{c}_{0}|\tilde{\vec{c}}_{k-t+1}]$.
\item Compute $z=c-\vec{c}.\vec{r}$ (mod $q$).
\end{enumerate}
~~Output $0$ if $|z|< q/4$ and 1 otherwise.
\paragraph{Correctness.} We just consider the case $H_{w}(id \& id^{'})=k\geq t$. $\tilde{\vec{c}}_{k-t+1}=\vec{c}_{k-t+1}.\mat{V}=\vec{s}^{T}\mat{B}\mat{V}+\vec{s}^{T}\mat{W}_{k-t+1}\mat{V}+\vec{e}^{T}_{k-t+1}\mat{V}$. It's easy to see that if $H_{w}(id \& id^{'})=k$, $\mat{W}_{k-t+1}\mat{V}=\mat{0}$, so $\tilde{\vec{c}}_{k-t+1}=\vec{s}^{T}\mat{B}\mat{V}+\vec{e}^{T}_{k-t+1}\mat{V}$. Then $\vec{c}=\vec{s}^{T}\mat{A}_{id}+[\vec{e}^{T}_{0}|\vec{e}^{T}_{k-t+1}\mat{V}]$, and $z=c-\vec{c}.\vec{r}$(mod $q$)$=\lfloor q/2 \rfloor m+\underbrace{e-[\vec{e}^{T}_{0}|\vec{e}^{T}_{k-t+1}\mat{V}]\vec{r}}_{\vec{e}}(mod\ q)$. If the error term $\vec{e}$ is small, our scheme is correct.
\paragraph{Security.} We sketch the proof of the Selective-ID security of the FIBE scheme described above. We have the following theorem:
\begin{theorem}
The FIBE scheme above is selectively secure provided that decision-$\LWE_{n,q,\chi}$ assumption holds.
\end{theorem}
\noindent $Proof$(sketch). We propose a sequence of games where the first game is identical to the real security game from the definition. In the last game in the sequence the adversary has advantage zero. We show that a $\ppt$ adversary $\A$ cannot distinguish between the games which will prove that the adversary has negligible advantage in winning the security game. The LWE problem is used in proving that $\mat{Game\ 3}$ and $\mat{Game\ 4}$ are indistinguishable.\\[0.2cm]
$\mat{Game\ 0}$. It's identical to the real game.\\[0.2cm]
$\mat{Game\ 1}$. We slightly change the way that the challenger generates $\pp$. Let $id^{*}=(x^{*}_{1},...,x^{*}_{N})$ be the challenge identity. The challenger chooses a random matrix $\mat{A} \in \Z^{n \times m}_{q}$, a random vector $\vec{u} \in \Z^{n}$, and chooses $\textbf{R}^{*}\in \{-1,1\}^{m \times m}$. Define the matrix
\begin{equation}
\mat{W}^{'}= \begin{bmatrix}
x^{*}_{1} \mat{I}_{n}&x^{*}_{2} \mat{I}_{n}&\ldots&x^{*}_{N} \mat{I}_{n}&-t \mat{I}_{n}
\end{bmatrix} \in \Z^{n \times \mu n}_{q},
\end{equation}
and set $\mat{B}=\mat{A}\mat{R}^{*}-\mat{W}^{'}.\G_{\mu n,l^{'},m} \in \Z^{n \times m}_{q}$. Outputs $\pp=(\mat{A}, \mat{B}, \vec{u})$, and sends $\pp$ to $\A$.\\[0.2cm]
\textbf{Game 2}. We change the way that the challenger answer the key queries. If the adversary $\A$ submits the key query for $id=(a_{1},...,a_{N})$, challenger first checks $H_{w}(id \& id^{'})<t$, if so, define the matrix
\begin{equation}
\mat{V}^{'}= \begin{bmatrix}
a_{1} \mat{I}_{n}\\
a_{2} \mat{I}_{n}\\
\vdots\\
a_{N} \mat{I}_{n}\\
\mat{I}_{n}
\end{bmatrix} \in \Z^{\mu n \times n}_{q},~~ \mat{V}=\G^{-1}_{\mu n,l^{'},m}(\mat{V}^{'}.\G_{n,2,m}).
\end{equation}
Set $\mat{U}=\mat{BV}, \mat{A}_{id}=[\mat{A}|\mat{U}]=[\mat{A}|\mat{A}\mat{R}^{*}\mat{V}-\mat{W}^{'}\mat{V}^{'}\G_{n,2,m}]$.\

Let $\vec{r}\leftarrow$ $\sampleright$ $(\mat{A},\mat{R}^{*}\mat{V},\mat{W}^{'}\mat{V}^{'}, \mat{T}_{\G_{n,2,m}}, \vec{u}, \sigma)$, so that $\mat{A}_{id}.\vec{r}=\vec{u}$. Outputs $\sk_{id}=\vec{r}$, and sends it to $\A$. Otherwise, outputs $\perp$, and aborts the game.\\[0.2cm]
$\mat{Game\ 3}$. Change the way to generate the challenge ciphertext. We use $\mat{R}^{*}$ as the matrix $\mat{R}_{1}$ used to generate $\vec{c}_{1}$, the remainder of the game is identical to $\mat{Game\ 2}$.\\[0.2cm]
$\mat{Game\ 4}$. We just choose the challenge ciphertext as a random independent element from the ciphertext space. So the advantage of $\A$ is zero.\\[0.2cm]
It easy to see in $\mat{Game\ 3}$, $\vec{c}_{0}=\vec{s}^{T}\mat{A}+\vec{e}^{T}_{0}$, $\vec{c}_{1}=(\vec{s}^{T}\mat{A}+\vec{e}^{T}_{0})\mat{R}^{*}$. $c=\vec{s}^{T}\vec{u}+e+\lfloor q/2\rfloor m$, $\vec{c}_{i}=(\vec{s}^{T}\mat{A}+\vec{e}^{T}_{0})\mat{R}^{*}+\vec{s}^{T}[\vec{0},...,\vec{0},-(i-1)\mat{I}_{n}]+\vec{e}^{T}_{i}-\vec{e}^{T}_{0}\mat{R}^{*}, i=2,...,N-t+1$. If $\vec{s}^{T}\mat{A}+\vec{e}^{T}_{0}$ and $\vec{s}^{T}\vec{u}+e$ are LWE instances, Then it simulates $\mat{Game\ 3}$, if they are random variables, it simulates $\mat{Game\ 4}$. \qed











\end{appendix}

\end{document}




\end{document} 